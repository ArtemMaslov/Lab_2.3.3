\textbf{\Large Описание эксперимента}

В работе измеряется осмотическое давление водного раствора жёлтой кровяной соли $K_4 Fe (CN)_6$, при нескольких значениях концентрации и проверяется справедливость закона Вант-Гоффа.

Молекулы жёлтой кровяной соли при растворении диссоциируют:
$$
K_4 Fe (CN)_6 \rightarrow 4 K^+ + [Fe (CN)_6]^{4-}
$$

Ионы $K^+$ свободно проникают через используемую в работе перегородку и не создают осмотическое давление.