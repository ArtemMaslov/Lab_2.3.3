\textbf{\Large Вывод}

В работе наблюдался прямой и обратный осмос, было измерено осмотическое давление при разной концентрации раствора.

\begin{tabular}{|c|c|c|}
	\hline
	$n$, $\%$ & $P_{осм}$, Па & $P_{выч}$, Па \\
	\hline
	$0,300$ & $9,6 \pm 0,6$  & $20,2$ \\
	$0,150$ & $7,5 \pm 3,0$  & $10,1$ \\
 	$0,075$ & $6,0 \pm 0,4$  & $5,0$  \\
	\hline
\end{tabular}

Проверить закон Вант-Гоффа не удалось, измеренное осмотическое давление не совпадает с вычисленным. Сложно определить, является ли зависимость $P_{осм}$ линейной, так как для этого нужно проводить измерения при большем числе концентраций.

На результат эксперимента могли повлиять следующие факторы:

\begin{enumerate}
	\item Во время измерений было обнаружено падение давления в системе на $\approx 7$ делений манометра. Течь устранить не удалось. Не герметичность системы могла повлиять на итоговые значения.
	\item Полупроницаемые перегородки давно не менялись и со временем могли засориться, что приводит к понижению измеренного давления от вычисленного по формуле Вант-Гоффа.
\end{enumerate}