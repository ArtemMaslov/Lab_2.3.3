\documentclass[14pt,a4paper]{extreport}
\usepackage[a4paper, left=2cm, right=2cm, top=1cm, bottom=2cm]{geometry}

\usepackage{listings} 
\usepackage{caption}
\usepackage{graphicx} % Для того, чтобы вставлять картинки.
\usepackage{wrapfig} % Картинка, обтекаемая текстом.
\usepackage{amssymb}
\usepackage{amsmath} % Чтобы вставлять обычный текст в формулу с помощью \text.

\usepackage{multicol} % Для написания текста в несколько колонок

\usepackage{float}   % Чтобы была опция таблиц H, запрещающая им бегать по документу
\restylefloat{table}

\usepackage{gensymb} % Геометрические символы. (градусы \degree)..

\usepackage[warn]{mathtext} % Русские символы в формулах. Нужно писать до пакета babel. Указывает, что в формулах используются символы кириллицы, которые по умолчанию печатаются прямым шрифтом.

\usepackage[T2A]{fontenc} % Установить кодировку шрифта для отображения кириллицы в формулах.
\usepackage[utf8]{inputenc}

\usepackage[russian]{babel} % Для переноса текста. Нельзя указывать одновременно russian и english, так как использует язык, который стоит правее.

\usepackage{indentfirst} % Красная строка в первом абзаце.

\usepackage{comment} % Для многострочный комментариев.
\setlength{\parindent}{1.25cm} % Отступ красной строки

\linespread{1.25} % Межстрочный интервал. По умолчанию 1.0

\DeclareSymbolFont{T2Aletters}{T2A}{cmr}{m}{it} % Сделать так, чтобы кириллица в формулах печаталась курсивом

% Объявляем новую команду для переноса строки внутри ячейки таблицы
\newcommand{\specialcell}[2][c]{%
	\begin{tabular}[#1]{@{}c@{}}#2\end{tabular}}

\begin{document}
	
	\begin{center}
		\large
		\textsc{Лабораторная работа №2.2.3}
		
		\LARGE
		\textbf{\textsc{Исследование осмотического давления}}
		\\[5mm]
		
		\large
		Маслов Артём\\
		Брицко Владимир\\
		Б01-104
		\\[3mm]
		18.05.2022
	\end{center}
	
	\textbf{\large Цель работы:}
	
	\begin{enumerate}
		\item Измерение осмотического давления при разной концентрации жёлтой кровяной соли;
		\item Проверка закона Вант-Гоффа.
	\end{enumerate}
	
	\textbf{\large Оборудование:}
	
	Осмометр; секундомер; пипетка; мерный стаканчик; химический стакан.
	
	\textbf{\Large Аннотация}

В работе изучаются свойства полупроницаемых перегородок. Измеряется осмотическое давления при разной концентрации кровяной соли. 

\textbf{\Large Теория}

\textit{Полупроницаемой перегородкой} называется перегородка, которая пропускает молекулы растворителя, но не пропускает молекулы растворённых в ней соединений. Прохождение растворителя через полупроницаемую перегородку называется \textit{осмосом}.


	
	\include{experimental scheme.tex}
	
	\textbf{\Large Описание эксперимента}

В работе измеряется осмотическое давление водного раствора жёлтой кровяной соли $K_4 Fe (CN)_6$, при нескольких значениях концентрации и проверяется справедливость закона Вант-Гоффа.

Молекулы жёлтой кровяной соли при растворении диссоциируют:
$$
K_4 Fe (CN)_6 \rightarrow 4 K^+ + [Fe (CN)_6]^{4-}
$$

Ионы $K^+$ свободно проникают через используемую в работе перегородку и не создают осмотическое давление.
	
	\textbf{\Large Вывод}

В работе наблюдался прямой и обратный осмос, было измерено осмотическое давление при разной концентрации раствора.

\begin{tabular}{|c|c|c|}
	\hline
	$n$, $\%$ & $P_{осм}$, Па & $P_{выч}$, Па \\
	\hline
	$0,300 \%$ & $9,6 \pm 0,6$  & $20,2$ \\
	$0,150 \%$ & $7,5 \pm 3,0$  & $10,1$ \\
 	$0,075 \%$ & $6,0 \pm 0,4$  & $5,0$  \\
	\hline
\end{tabular}

Проверить закон Вант-Гоффа не удалось, измеренное осмотическое давление не совпадает с вычисленным. Сложно определить, является ли зависимость $P_{осм}$ линейной, так как для этого нужно проводить измерения при большем числе концентраций.

На результат эксперимента могли повлиять следующие факторы:

\begin{enumerate}
	\item Во время измерений было обнаружено падение давления в системе на $\approx 7$ делений манометра.
	\item Полупроницаемые перегородки давно не менялись и со временем могли засориться, что приводит к понижению измеренного давления от вычисленного по формуле Вант-Гоффа. 
\end{enumerate}
	
\end{document}